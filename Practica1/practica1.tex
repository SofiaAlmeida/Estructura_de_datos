\documentclass[titlepage, 12pt,a4paper]{article}

% Packages
\usepackage[utf8]{inputenc}						%acentos y mierda
\usepackage[spanish, es-tabla]{babel}			%español
\usepackage{caption}							%fotos cortandose
\usepackage{listings}							%meter codigo 
\usepackage{adjustbox}							%coin no se acordaba
\usepackage{enumitem}							%listas mejoradas
\usepackage{amssymb, amsmath, amsthm}			%Los de matemáticas
\usepackage[margin=1in, top=0.7in]{geometry}	%Paquete de los margenes
\usepackage{xcolor}								%para definir tus propios colores
\usepackage{soul}								%coin no se acordaba
\usepackage{graphicx}
\graphicspath{ {/Estructura_de_datos/Images/} }
% Meta

\vspace{2cm}
\author{Pedro Bonilla Nadal, Sofía Almeida Bruno, Jesús Sáchez de Lechina Tejada}
\date{\small{}}

% Custom  
\providecommand{\abs}[1]{\lvert#1\rvert}
\setlength\parindent{0pt}
\definecolor{Light}{gray}{.90}
\newcommand\ddfrac[2]{\frac{\displaystyle #1}{\displaystyle #2}}
\setlist[description]{leftmargin=1em, labelindent=0.5em}

\begin{document}
\begin{titlepage}
	\vspace{1cm}
	\centering
	{\small II Doble Grado Ingieniería Infomática y Matemáticas  \par}
	\vspace{3.5cm}
	{\huge\bfseries  Estructura de Datos\par}
	\vspace{2.5cm}
	{\scshape\Large Práctica 1 - Eficiencia\par}
	\vspace{2cm}
	{\Large\itshape Pedro Bonilla Nadal, Sofía Almeida Bruno,  \par}
	{\Large\itshape Jesús Sánchez de Lechina Tejada \par}
	\vfill

	\vfill

% Bottom of the page
	{\large \par}
\end{titlepage}

	\textbf{\large Ejercicio 1.} El siguiente código realiza la ordenación mediante el algoritmo de la burbuja:\vspace {1em}
	
	\hspace*{1cm}1. void ordenar(int *v, int n) \{ \\
	\hspace*{1cm}2.\hspace*{2em}	for (int i=0; i<n-1; i++)\\
	\hspace*{1cm}3.\hspace*{4em}		for (int j=0; j $<$ n-i-1; j++) \{ \\
	\hspace*{1cm}4.\hspace*{6em}			if (v[j] $>$ v[j+1]) \{\\
	\hspace*{1cm}5.\hspace*{8em}				int aux = v[j];\\
	\hspace*{1cm}6.\hspace*{8em}				v[j] = v[j+1];\\
	\hspace*{1cm}7.\hspace*{8em}				v[j+1] = aux;\\
	\hspace*{1cm}8.\hspace*{6em}\} \\
	\hspace*{1cm}9.\hspace*{4em}\} \\
	\hspace*{1cm}10. \}  \vspace {1em} \\
Calcule la eficiencia teórica de este algoritmo. A continuación replique el experimento que se ha hecho antes (búsqueda lineal) con este nuevo código. Debe:\\ 
	\begin{itemize}  
	\item Crear un fichero ordenacion.cpp con el programa completo para realizar una
ejecución del algoritmo.
	\item Crear un fichero ejecuciones\_ordenación.csh que permite ejecutar varias veces el programa anterior y generar un fichero con los datos obtenidos. 
	\item Usar gnuplot para dibujar los datos obtenidos en el apartado previo.
	\end{itemize}
Los datos deben contener tiempos de ejecución para tamaños del vector 100, 600, 1100, \ldots,  30000. Pruebe a dibujar superpuestas la función con la eficiencia teórica y la empírica. ¿Qué sucede?\\

\underline{\emph{Solución.}}\\
El archivo ordenación.cpp y ejecuciones\_ordenacion.csh se encuentran adjuntos en la práctica.\\

\underline{Eficiencia teórica:} \\
	Línea 2: 5OE (2 asignaciones, 3 operaciones aritmético-lógica).\\
	Línea 3: 6OE (2 asignaciones, 4 operaciones aritmético-lógica).\\
	Línea 4-7: 13OE (6 accesos a vector, 4 operaciones artimético-lógicas, 3 asignaciones).\\

Entonces:= 1 + $\sum_{i=0}^{n-2}(5+\sum_{j=0}^{n-i-2}(5+13))= 1 + \sum_{i=0}^{n-2}(5+((n-i-2)*18))= \\
1 + \sum_{i=0}^{n-2}(5) + \sum_{i=0}^{n-2}(18n)+\sum_{i=0}^{n-2}(18n)+ \sum_{i=0}^{n-2}(18i) + \sum_{i=0}^{n-2}(18) = -4 + 4n + 9 n^2$.\\

\small{\underline{Nota:} debemos considerar que para ordenar un vector debe tener al menos 2 elementos.}

	


\end{document}
